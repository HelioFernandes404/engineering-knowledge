%-------------------------
% Cover Letter in Latex
% Author: Hélio Fernandes
% Based on CV template structure
% License: MIT
%------------------------

\documentclass[letterpaper,11pt]{article}

\usepackage{latexsym}
\usepackage[empty]{fullpage}
\usepackage{titlesec}
\usepackage{marvosym}
\usepackage[usenames,dvipsnames]{color}
\usepackage{verbatim}
\usepackage{enumitem}
\usepackage[hidelinks]{hyperref}
\usepackage{fancyhdr}
\usepackage[english]{babel}
\usepackage{tabularx}

% Font - matching CV
\usepackage[sfdefault]{noto-sans}

\pagestyle{fancy}
\fancyhf{} % clear all header and footer fields
\fancyfoot{}
\renewcommand{\headrulewidth}{0pt}
\renewcommand{\footrulewidth}{0pt}

% Adjust margins
\addtolength{\oddsidemargin}{-0.5in}
\addtolength{\evensidemargin}{-0.5in}
\addtolength{\textwidth}{1in}
\addtolength{\topmargin}{-.5in}
\addtolength{\textheight}{1.0in}

\urlstyle{same}

\raggedbottom
\raggedright
\setlength{\tabcolsep}{0in}

%-------------------------------------------
%%%%%%  COVER LETTER STARTS HERE  %%%%%%%%%%%%%%%%%%%%%%%%%%%%

\begin{document}

%----------HEADING----------
\begin{center}
    \textbf{\Huge \scshape Hélio Fernandes} \\ \vspace{1pt}
    \small +5587988023661 $|$ \href{mailto:heliofer25@icloud.com}{\underline{heliofer25@icloud.com}} $|$
    \href{https://www.linkedin.com/in/heliofernandes/}{\underline{linkedin.com/in/heliofernandes}} $|$
    \href{https://github.com/HelioFernandes404}{\underline{github.com/HelioFernandes404}}
\end{center}

\vspace{10pt}

%----------DATE AND RECIPIENT----------
\noindent
\today

\vspace{10pt}

\noindent
Equipe de Recrutamento \\
DAI-TEC \\
Brasil

\vspace{10pt}

%----------OPENING----------
\noindent
Prezados recrutadores,

\vspace{10pt}

%----------BODY----------
\noindent
É com grande entusiasmo que me candidato à posição de \textbf{Engineer (Pleno) Machine Learning / Gen AI} na DAI-TEC. Como ex-colaborador que atuou na empresa entre 2019 e 2023, tenho profundo conhecimento da cultura, valores e desafios técnicos da organização, e agora retorno com habilidades aprimoradas e experiência adicional em infraestrutura de IA em escala empresarial.

\vspace{10pt}

\noindent
Durante minha passagem anterior na DAI-TEC, tive a oportunidade de desenvolver um \textbf{ecossistema completo de IA conversacional com impacto mensurável em eficiência operacional}, onde implementei arquiteturas RAG (Retrieval-Augmented Generation) utilizando LlamaIndex e OpenAI API. Essa experiência me proporcionou domínio prático em orquestração de modelos generativos, chunking estratégico, similarity search com embeddings, e desenvolvimento de agentes inteligentes aplicados a casos reais de negócio. Construí arquiteturas de microsserviços containerizados que incluíam sistemas de cobrança (Stripe), plataformas de mensageria (WhatsApp), e bancos de dados vetoriais (Qdrant), demonstrando minha capacidade de entregar soluções end-to-end escaláveis com foco em ROI.

\vspace{10pt}

\noindent
Após minha saída da DAI-TEC, passei os últimos anos na SystemFrame como Software Engineer, onde \textbf{aprofundei minhas competências em infraestrutura cloud e MLOps}. Gerenciei infraestrutura como código usando Terraform e Ansible, automatizei deployments em Kubernetes com Helm, e implementei pipelines de CI/CD robustos em AWS. Essa experiência complementa perfeitamente os requisitos da vaga, pois desenvolvi expertise em plataformas de nuvem (AWS), ferramentas de processamento de dados (Python/PySpark), e integração de sistemas corporativos em ambientes de produção de alta disponibilidade.

\vspace{10pt}

\noindent
Minha vivência técnica alinha-se diretamente com os requisitos da posição:

\vspace{5pt}

\begin{itemize}[leftmargin=0.2in, itemsep=2pt]
    \item \textbf{LLMs e Orquestração de Agentes}: Experiência prática com OpenAI API, arquiteturas RAG (chunking, embeddings, similarity search), e frameworks modernos (LlamaIndex). Preparado para trabalhar com LangChain/LangGraph e implementar soluções multiagente escaláveis.
    \item \textbf{Pipelines de Dados para ML}: Desenvolvimento de pipelines completos de ingestão, pré-processamento, embedding e deploy usando Python, Docker, e AWS. Domínio de SQL, processamento distribuído (PySpark), e bancos vetoriais (Qdrant) com estratégias de retrieval otimizadas.
    \item \textbf{Cloud, MLOps e Observabilidade}: Sólida experiência em AWS (S3, Lambda, SageMaker, DynamoDB), infraestrutura como código (Terraform), CI/CD para ML, e práticas de DevOps (Kubernetes, Helm, monitoramento com Prometheus/Grafana). Familiarizado com práticas de avaliação e guardrails para LLMs.
    \item \textbf{Integração de Sistemas e APIs}: Histórico comprovado integrando sistemas legados e modernos, desenvolvendo APIs REST/GraphQL, e conectando múltiplos serviços em arquiteturas de microsserviços distribuídas.
\end{itemize}

\vspace{10pt}

\noindent
Além das competências técnicas, trago uma \textbf{compreensão profunda do contexto de negócio da DAI-TEC} e a capacidade de colaborar efetivamente com times multidisciplinares (marketing, vendas, operações) para identificar oportunidades de inovação via IA com impacto mensurável. Estou alinhado com práticas modernas de segurança e compliance em LLMs (guardrails, validação de outputs, bias mitigation), e possuo inglês técnico para leitura de documentação, papers e participação em comunidades internacionais de IA.

\vspace{10pt}

\noindent
Retornar à DAI-TEC representa para mim a oportunidade de contribuir com um time que já conheço e admiro, agora com uma bagagem técnica significativamente expandida em arquiteturas de IA Generativa, práticas de MLOps, e frameworks emergentes de orquestração. Estou confiante de que minha combinação única de \textbf{experiência prévia na empresa + expertise comprovada em RAG, infraestrutura cloud e deploy de soluções de ML em produção} me posiciona como um candidato diferenciado para entregar impacto imediato nesta vaga.

\vspace{10pt}

%----------CLOSING----------
\noindent
Agradeço a atenção e coloco-me à disposição para discutir como posso contribuir para os desafios de IA/ML da DAI-TEC.

\vspace{10pt}

\noindent
Atenciosamente,

\vspace{5pt}

\noindent
\textbf{Hélio Fernandes}

%-------------------------------------------
\end{document}
